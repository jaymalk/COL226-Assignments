\documentclass{article}
\usepackage[utf8]{inputenc}
\usepackage[english]{babel}

\title{Assignment 1} 
\author{Rajbir Malik}
\date{2017CS10416}

\begin{document}

\maketitle
\subsection*{Theorem}

\begin{center}
\(\forall\) \texttt{exptree} \textbf{e} \\ \texttt{mk\_big }(\texttt{eval(\textbf{e})}) \(=\) \texttt{stackmc li (compile \textbf{e})} \\
\end{center}

\subsection*{Proof}
We shall be proving the theorem by applying induction on the \texttt{height} of \texttt{exptree}. \\
\subsubsection*{Base Case}
The base case \texttt{height} for \texttt{exptree} is 0. In this case the tree \texttt{e}, is just \texttt{N(x)} for some \texttt{int}.
In this case,
\begin{center}
    \texttt{eval(e)} \(=\) \texttt{x} and,\\
    \texttt{compile(e)} \(=\) \texttt{[CONST(mk\_big x)]} \\
\end{center}
    which after evaluation with stackmc would give
\begin{center}
    \texttt{stackmc li (compile e)} = (mk\_big x)
\end{center}
Thus, with this we can conclude
\begin{center}
    \texttt{mk\_big }(\texttt{eval(\textbf{e})}) \(=\) \texttt{stackmc li (compile \textbf{e})}
\end{center}
\phantom \\
\subsubsection*{Inductive Hypothesis}
Now, we assume that \( \forall \) \texttt{exptree e} with \texttt{height(e)} \(\leq k\) satisfy the above theorem, i.e. \texttt{mk\_big }(\texttt{eval(\textbf{e})}) \(=\) \texttt{stackmc li (compile \textbf{e})}, where \( k \geq 0\). \\[10 pt]
\subsubsection*{Inductive Step}
Now, let \texttt{e} be an \texttt{exptree} with \texttt{height} \(= k+1 \).\\[5pt]Now, since  \( k \geq 0\), \texttt{height} of \texttt{e} \( \geq 1\).\\[5pt] This ensures that \texttt{e} is of the form 
\begin{center}
    \texttt{(BIN of exptree*exptree)}\\ or \\\texttt{(UN of exptree)}
\end{center} 
, where \texttt{BIN and UN} are binary and unary operations respectively. \\[10 pt]
\textbf{Case 1 (Binary Operation)} \\ [5 pt]
    Now, \texttt{e} is of the form \texttt{BIN(el, er)}. Since, \texttt{el and er} are subtrees of \texttt{e} their \texttt{height} is going to be \( \leq k\). Thus, our induction hypothesis holds for these trees. Therefore,
    \begin{center}
        \texttt{mk\_big }(\texttt{eval(\textbf{er})}) \(=\) \texttt{stackmc li (compile \textbf{er})} \(=\) \texttt{mk\_big xr} (say) \\
        \texttt{mk\_big }(\texttt{eval(\textbf{el})}) \(=\) \texttt{stackmc li (compile \textbf{el})} \(=\) \texttt{mk\_big xl} (say)
    \end{center}
    Also,
    \begin{center}
        \texttt{compile(e)} = \texttt{compile(el) $@$ compile(er) $@$ [BIN]}   
    \end{center}
    Now, for \texttt{LHS}, 
    \begin{center}
        \texttt{eval(e)} = \texttt{eval(el) ** eval(er)}, and so \\
        \texttt{eval(e)} = \texttt{xl ** xr}                            
    \end{center}
    where ** is the syntactic representation of \texttt{BIN}. \\[5pt]
    Now, for \texttt{RHS} 
    \begin{center}
        {\small \texttt{stackmc l1 (compile e)} = \texttt{stackmc l1 (compile el)$@$(compile er)$@$[BIN]}}
    \end{center}
    And, by the definition of stackmc, since \texttt{compile el} represents a complete tree, its values \texttt{mk\_big xl} will be prepended to stack. The same goes for \texttt{compile er}. And, therefore,
    \begin{center}
        {\small \texttt{stackmc l1 (compile e)} = \texttt{stackmc ((mk\_big xr)::(mk\_big xl)::l1) [BIN]}}
    \end{center}
    Now, following the definition of \texttt{stackmc}, the above expression evaluates to,
    \begin{center}
        {\texttt{stackmc l1 (compile e)} = \texttt{BIN (mk\_big xl) (mk\_big xr)}}, {\tiny which is same as} \\[5pt]
        {\small \texttt{BIN (mk\_big xl) (mk\_big xr)} = \texttt{mk\_big (xl ** xr)} = \texttt{mk\_big (eval(e)}}
    \end{center}
    and, hence induction holds. \\
\pagebreak
\\
\textbf{Case 2 (Unary Operation)} \\ [5 pt]
With most properties same as above and \texttt{e} as \texttt{UN(e')}, by induction hypothesis,
\begin{center}
    \texttt{mk\_big eval(e')} =  \texttt{stackmc li (compile \textbf{e'})} \(=\) \texttt{mk\_big x'} (say) 
\end{center}
Also,
    \begin{center}
        \texttt{compile(e)} = \texttt{compile(e') $@$ [UN]}
    \end{center}
    Now, for \texttt{LHS}, 
    \begin{center}
        \texttt{eval(e)} = \texttt{ \#eval(e')},\\
        \texttt{eval(e)} = \texttt{ \#x}                               
    \end{center}
    where \# is the syntactic representation of \texttt{UN}. \\[5pt]
    Now, for \texttt{RHS} 
    \begin{center}
        {\small \texttt{stackmc l1 (compile e)} = \texttt{stackmc l1 (compile e')$@$[BIN]}}
    \end{center}
    And, by the definition of stackmc, since \texttt{compile e'} represents a complete tree, its values \texttt{mk\_big x'} will be prepended to stack. And, therefore,
    \begin{center}
        {\small \texttt{stackmc l1 (compile e)} = \texttt{stackmc ((mk\_big x')::l1) [BIN]}}
    \end{center}
    Now, following the definition of \texttt{stackmc}, the above expression evaluates to,
    \begin{center}
        {\texttt{stackmc l1 (compile e)} = \texttt{UN (mk\_big x')}}, {\tiny which is same as} \\[5pt]
        {\small \texttt{UN (mk\_big x')} = \texttt{mk\_big (\#xr)} = \texttt{mk\_big (eval(e)}}
    \end{center}
    and, hence induction holds. \\

{\large Thus, induction holds, and thus the theorem is correct.}

\subsubsection*{Note}
We have used some properties which are cleared here
\begin{itemize}
    \item \texttt{UN (mk\_big x')} = \texttt{mk\_big (\#xr)}, from definition of \texttt{UN} in \texttt{bigint}
    \item \texttt{BIN (mk\_big xl) (mk\_big xr)} = \texttt{mk\_big (xl ** xr)},  from definition of \texttt{BIN} in \texttt{bigint}
\end{itemize}
\bibliographystyle{plain}

\bibliography{references}
\end{document}