\documentclass{article}
\usepackage[utf8]{inputenc}
\usepackage[english]{babel}

\usepackage[dvipsnames]{xcolor}


\title{Assignment 1}
\author{Rajbir Malik}
\date{2017CS10416}

\begin{document}

\maketitle
\subsection*{Theorem}

\begin{center}
\(\forall\) \texttt{\textcolor{Maroon}{exptree}} \textbf{e} \\[5pt] \texttt{\textcolor{Salmon}{mk\_big} }(\texttt{\textcolor{Fuchsia}{eval}(\textbf{e})}) \(=\) \texttt{\textcolor{Fuchsia}{stackmc} li (\textcolor{Fuchsia}{compile} \textbf{e})} \\
\end{center}

\subsection*{Proof}
We shall be proving the theorem by applying induction on the \texttt{\textcolor{Fuchsia}{height}} of \texttt{\textcolor{Maroon}{exptree}}. \\
\subsubsection*{Base Case}
The base case \texttt{\textcolor{Fuchsia}{height}} for \texttt{\textcolor{Maroon}{exptree}} is 0. In this case the tree \texttt{e}, is just \texttt{N(x)} for some \texttt{int}.
In this case,
\begin{center}
    \texttt{\textcolor{Fuchsia}{eval}(e)} \(=\) \texttt{x}, and\\[5pt]
    \texttt{\textcolor{Fuchsia}{compile}(e)} \(=\) \texttt{[CONST(\textcolor{Salmon}{mk\_big} x)]} \\
\end{center}
    which after evaluation with stackmc would give
\begin{center}
    \texttt{\textcolor{Fuchsia}{stackmc} li (\textcolor{Fuchsia}{compile} e)} = (\textcolor{Salmon}{mk\_big} x)
\end{center}
Thus, with this we can conclude
\begin{center}
    \texttt{\textcolor{Salmon}{mk\_big} }(\texttt{\textcolor{Fuchsia}{eval}(\textbf{e})}) \(=\) \texttt{\textcolor{Fuchsia}{stackmc} li (\textcolor{Fuchsia}{compile} \textbf{e})}
\end{center}
\phantom \\
\subsubsection*{Inductive Hypothesis}
Now, we assume that \( \forall \) \texttt{\textcolor{Maroon}{exptree} e} with \texttt{\textcolor{Fuchsia}{height}(e)} \(\leq k\) the above theorem is satisfied, i.e. \texttt{\textcolor{Salmon}{mk\_big} }(\texttt{\textcolor{Fuchsia}{eval}(\textbf{e})}) \(=\) \texttt{\textcolor{Fuchsia}{stackmc} li (\textcolor{Fuchsia}{compile} \textbf{e})}, where \( k \geq 0\). \\[10 pt]
\subsubsection*{Inductive Step}
Now, let \texttt{e} be an \texttt{\textcolor{Maroon}{exptree}} with \texttt{\textcolor{Fuchsia}{height}} \(= k+1 \).\\[5pt]Now, since  \( k \geq 0\), \texttt{\textcolor{Fuchsia}{height}} of \texttt{e} \( \geq 1\).\\[5pt] This ensures that \texttt{e} is of the form
\begin{center}
    \texttt{(\textcolor{blue}{BIN} of \textcolor{Maroon}{exptree}*\textcolor{Maroon}{exptree})}\\ or \\\texttt{(\textcolor{blue}{UN} of \textcolor{Maroon}{exptree})}
\end{center}
, where \texttt{\textcolor{blue}{BIN} and \textcolor{blue}{UN}} are \textit{binary and unary} operations respectively. \\[10 pt]
\textbf{Case 1 (Binary Operation)} \\ [5 pt]
    Now, \texttt{e} is of the form \texttt{\textcolor{blue}{BIN}(el, er)}. Since, \texttt{el and er} are subtrees of \texttt{e} their \texttt{\textcolor{Fuchsia}{height}} is going to be \( \leq k\). Thus, our induction hypothesis holds for these trees. Therefore,
    \begin{center}
        \texttt{\textcolor{Salmon}{mk\_big} }(\texttt{\textcolor{Fuchsia}{eval}(\textbf{er})}) \(=\) \texttt{\textcolor{Fuchsia}{stackmc} li (\textcolor{Fuchsia}{compile} \textbf{er})} \(=\) \texttt{\textcolor{Salmon}{mk\_big} xr} (say) \\[5pt]
        \texttt{\textcolor{Salmon}{mk\_big} }(\texttt{\textcolor{Fuchsia}{eval}(\textbf{el})}) \(=\) \texttt{\textcolor{Fuchsia}{stackmc} li (\textcolor{Fuchsia}{compile} \textbf{el})} \(=\) \texttt{\textcolor{Salmon}{mk\_big} xl} (say)
    \end{center}
    Also,
    \begin{center}
        \texttt{\textcolor{Fuchsia}{compile}(e)} = \texttt{\textcolor{Fuchsia}{compile}(el) $@$ \textcolor{Fuchsia}{compile}(er) $@$ [\textcolor{blue}{BIN}]}
    \end{center}
    Now, for \texttt{LHS},
    \begin{center}
        \texttt{\textcolor{Fuchsia}{eval}(e)} = \texttt{\textcolor{Fuchsia}{eval}(el) ** eval(er)}, and so \\
        \texttt{\textcolor{Fuchsia}{eval}(e)} = \texttt{xl ** xr}
    \end{center}
    where ** is the syntactic representation of \texttt{\textcolor{blue}{BIN}}. \\[5pt]
    Now, for \texttt{RHS}
    \begin{center}
        {\small \texttt{\textcolor{Fuchsia}{stackmc} l1 (\textcolor{Fuchsia}{compile} e)} = \texttt{\textcolor{Fuchsia}{stackmc} l1 (\textcolor{Fuchsia}{compile} el)$@$(\textcolor{Fuchsia}{compile} er)$@$[\textcolor{blue}{BIN}]}}
    \end{center}
    And, by the definition of stackmc, since \texttt{\textcolor{Fuchsia}{compile} el} represents a complete tree, its values \texttt{\textcolor{Salmon}{mk\_big} xl} will be prepended to stack. The same goes for \texttt{\textcolor{Fuchsia}{compile} er}. And, therefore,
    \begin{center}
        {\small \texttt{\textcolor{Fuchsia}{stackmc} l1 (\textcolor{Fuchsia}{compile} e)} = \texttt{\textcolor{Fuchsia}{stackmc} ((\textcolor{Salmon}{mk\_big} xr)::(\textcolor{Salmon}{mk\_big} xl)::l1) [\textcolor{blue}{BIN}]}}
    \end{center}
    Now, following the definition of \texttt{\textcolor{Fuchsia}{stackmc}}, the above expression evaluates to,
    \begin{center}
        {\texttt{\textcolor{Fuchsia}{stackmc} l1 (\textcolor{Fuchsia}{compile} e)} = \texttt{\textcolor{blue}{BIN} (\textcolor{Salmon}{mk\_big} xl) (\textcolor{Salmon}{mk\_big} xr)}}, {\tiny which is same as} \\[5pt]
        {\small \texttt{\textcolor{blue}{BIN} (\textcolor{Salmon}{mk\_big} xl) (\textcolor{Salmon}{mk\_big} xr)} = \texttt{\textcolor{Salmon}{mk\_big} (xl ** xr)} = \texttt{\textcolor{Salmon}{mk\_big} (eval(e)}}
    \end{center}
    and, hence induction holds. \\
\pagebreak
\\
\textbf{Case 2 (Unary Operation)} \\ [5 pt]
With most properties same as above and \texttt{e} as \texttt{\textcolor{blue}{UN}(e')}, by induction hypothesis,
\begin{center}
    \texttt{\textcolor{Salmon}{mk\_big} eval(e')} =  \texttt{\textcolor{Fuchsia}{stackmc} li (\textcolor{Fuchsia}{compile} \textbf{e'})} \(=\) \texttt{\textcolor{Salmon}{mk\_big} x'} (say)
\end{center}
Also,
    \begin{center}
        \texttt{\textcolor{Fuchsia}{compile}(e)} = \texttt{\textcolor{Fuchsia}{compile}(e') $@$ [\textcolor{blue}{UN}]}
    \end{center}
    Now, for \texttt{LHS},
    \begin{center}
        \texttt{\textcolor{Fuchsia}{eval}(e)} = \texttt{ \#eval(e')},\\[5pt]
        \texttt{\textcolor{Fuchsia}{eval}(e)} = \texttt{ \#x}
    \end{center}
    where \# is the syntactic representation of \texttt{\textcolor{blue}{UN}}. \\[5pt]
    Now, for \texttt{RHS}
    \begin{center}
        {\small \texttt{\textcolor{Fuchsia}{stackmc} l1 (\textcolor{Fuchsia}{compile} e)} = \texttt{\textcolor{Fuchsia}{stackmc} l1 (\textcolor{Fuchsia}{compile} e')$@$[\textcolor{blue}{BIN}]}}
    \end{center}
    And, by the definition of stackmc, since \texttt{\textcolor{Fuchsia}{compile} e'} represents a complete tree, its values \texttt{\textcolor{Salmon}{mk\_big} x'} will be prepended to stack. And, therefore,
    \begin{center}
        {\small \texttt{\textcolor{Fuchsia}{stackmc} l1 (\textcolor{Fuchsia}{compile} e)} = \texttt{\textcolor{Fuchsia}{stackmc} ((\textcolor{Salmon}{mk\_big} x')::l1) [\textcolor{blue}{BIN}]}}
    \end{center}
    Now, following the definition of \texttt{\textcolor{Fuchsia}{stackmc}}, the above expression evaluates to,
    \begin{center}
        {\texttt{\textcolor{Fuchsia}{stackmc} l1 (\textcolor{Fuchsia}{compile} e)} = \texttt{\textcolor{blue}{UN} (\textcolor{Salmon}{mk\_big} x')}}, {\tiny which is same as} \\[5pt]
        {\small \texttt{\textcolor{blue}{UN} (\textcolor{Salmon}{mk\_big} x')} = \texttt{\textcolor{Salmon}{mk\_big} (\#xr)} = \texttt{\textcolor{Salmon}{mk\_big} (eval(e)}}
    \end{center}
    and, hence induction holds. \\

{\large Thus, induction holds, and thus the theorem is correct.}

\subsubsection*{}
\subsubsection*{}
\section*{}
\subsubsection*{Note}
We have used some properties which are cleared here
\begin{itemize}
    \item \texttt{\textcolor{blue}{UN} (\textcolor{Salmon}{mk\_big} x')} = \texttt{\textcolor{Salmon}{mk\_big} (\#xr)}, from definition of \texttt{\textcolor{blue}{UN}} in \texttt{bigint}
    \item \texttt{\textcolor{blue}{BIN} (\textcolor{Salmon}{mk\_big} xl) (\textcolor{Salmon}{mk\_big} xr)} = \texttt{\textcolor{Salmon}{mk\_big} (xl ** xr)},  from definition of \texttt{\textcolor{blue}{BIN}} in \texttt{bigint}
    \item List \texttt{li} is assumed to be an arbitrary \texttt{bigint list}.
\end{itemize}
\bibliographystyle{plain}

\bibliography{references}
\end{document}
